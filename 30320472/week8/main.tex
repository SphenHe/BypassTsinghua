\documentclass{article}
\usepackage[UTF8]{ctex}  % 使用中文支持包
\usepackage[a4paper, margin=1in]{geometry}  % 设置纸张大小和边距
\usepackage{anyfontsize}  % 解决字体大小报错问题
\usepackage{fancyhdr}  % 设置页眉、页脚、页码
\usepackage{longtable}  % 支持长表格

\usepackage{amsmath}  % 数学公式支持
\usepackage{cases}  % 支持联立编号
\usepackage{cite}  % 引用支持

\usepackage{graphicx}  % 插入图片支持
\usepackage{float}  % 设置图片浮动位置
\usepackage{subfigure}  % 插入多图时用子图显示

\usepackage{listings}  % 代码块支持
\usepackage{xcolor}  % 设置代码块颜色

\usepackage[hyphens]{url}  % 支持链接换行
\usepackage{hyperref}  % 超链接支持
\usepackage{lastpage}  % 添加lastpage包

\usepackage{gbt7714}  %国标参考文献

\hypersetup{
    hidelinks,
    colorlinks=true,
    allcolors=black,
    pdfstartview=Fit,
    breaklinks=true
}

\title{聚变能源概论-第七讲作业}
\author{\LaTeX\ by\ Jerry\ }
\date{\today}
\pagenumbering{arabic}

\begin{document}
\pagestyle{fancy}

\fancyhead[L]{Jerry}
\fancyhead[C]{聚变能源概论-第七讲作业}
\fancyhead[R]{\today}
\fancyfoot[C]{Page \thepage/\pageref{LastPage}}

\section*{6.1}

\emph{1. 一简单磁镜的$B_{\min}=3T,B_{\max}=5T$,氘等离子体的温度$T=10keV$,密度$n=5\times10^{19}m^{-3}$。试估计该装置的粒子损失速率(提示:考虑碰撞过程)。}

离子-离子的碰撞频率为$$\nu_{ii} = \frac{e^4 \ln{\Lambda}}{6\sqrt{3}\pi\varepsilon_0^2} \cdot \frac{n_i Z^4}{m_i^{1/2} T_i^{3/2}} = 89 \text{s}^{-1}$$

即 $$\tau_{ii} = \frac{1}{\nu_{ii}} = 0.011 \text{s}$$

损失过程的时间常数为$$\tau_{\text{loss}} = \frac{\tau_{ii}}{k} = 0.031 \text{s}$$

磁镜有损失锥的粒子不受约束会跑掉,其比例为$$k = 2 \frac{\int_{0}^{2\pi} d\varphi \int_{0}^{\theta_m} \sin \theta d\theta}{4\pi} = 1 - \cos \theta_m = 1 - \sqrt{1 - \frac{B_{\text{min}}}{B_{\text{max}}}} = 0.368$$

故损失率为$$\frac{\text{d}n}{\text{d}t} = -\frac{n}{\tau_{\text{loss}}} = 1.6 \times 10^{21} \text{m}^{-3}\text{s}^{-1}$$

\section*{6.3}

\emph{2.(1)推导含有z 向磁场的 Z 簇缩装置的平衡方程式(6.10);(2)推导等离子体柱之外的 z 向磁场等于0时腊肠不稳定性(m=0)及弯曲不稳定性(m=1)的稳定条件;(3)说明z 向磁场对稳定性的影响。}

(1):

由磁流体力学静态平衡方程$ j \times B= \nabla p $和$\nabla \times B= \mu_0 j$ ,使用矢量公式$\nabla.(a\cdot b)=(b\cdot\nabla)a+(a\cdot\nabla)b+b\times(\nabla\times a)+a\times(\nabla\times b)$, 得$$\nabla p=\frac{1}{\mu_0}(\nabla\times B)\times B \Rightarrow \nabla p=\frac{1}{\mu_0}\left((B\cdot \nabla)B-\frac{\nabla B^2}{2}\right) \Rightarrow \nabla\left(p+\frac{B^2}{2\mu_0}\right)=\frac{1}{\mu_0}(B\cdot \nabla)B$$

Z 轴缩装置轴对称,$\frac{\partial}{\partial\theta}=\frac{\partial}{\partial z}=0$,$\nabla p$只有$r$分量,由讲义6.2知磁场$B$没有$r$分量,则平衡方程的径向$r$分量: $$(\nabla f)_r=\frac{\partial f}{\partial r},(A\cdot \nabla B)_r=A_r\frac{\partial B_r}{\partial r}+\frac{A\phi_r}{r}\frac{\partial B_r}{\partial \phi}+A_z\frac{\partial B_r}{\partial z}-\frac{A\phi B_\phi}{r}$$

柱坐标系下:$$\frac{\partial}{\partial r}\left(p+\frac{B_o^2+B_z^2}{2\mu_0}\right)=-\frac{B_{\phi}^2}{r\mu_0}$$

得到轴对称系统一般平衡方程$$\frac{\partial}{\partial r}\left(p+\frac{B_o^2}{2\mu_0}\right)=-\frac{B_{\phi}^2}{r\mu_0}$$

对于纯的Z箍缩,$B_z=0$,则得到$$\frac{\partial}{\partial r}\left(p+\frac{B_o^2}{2\mu_0}\right)=-\frac{B_{\phi}^2}{r\mu_0}$$

对$r^2\frac{\partial}{\partial r}\left(p+\frac{B_o^2}{2\mu_0}\right)=-r^2\frac{B_{\phi}^2}{r\mu_0}$进行积分,得到$$\left(p+\frac{B_{\theta}^{2}}{2\mu_0}\right){|r=a}=\frac{1}{\pi a^2}\int{0}^{a}p \cdot 2\pi rdr \Rightarrow p(a)+\frac{B_{\theta}^{2}(a)}{2\mu_0}=\langle p\rangle$$

由于边界真空条件$p(a)=0$,得到$$\frac{B_{\theta}^{2}(a)}{2\mu_0}=\langle p\rangle$$

(2):

对于$m=0$的腊肠不稳定性,只要半径缩小时,内部磁压升高得比外部磁压多,就是稳定的。假设半径缩小$\delta r$

$$\delta B_\theta^2 = B_{\theta 0}^2 r^2 \left( \frac{1}{(r - \delta r)^2} - \frac{1}{r^2} \right) B_{\theta 0}^2 \frac{2\delta r}{r} $$

$$\delta B_z^2 = B_{z 0}^2 r^2 \left( \frac{1}{(r - \delta r)^2} - \frac{1}{r^2} \right) B_{z 0}^2 \frac{4\delta r}{r} $$

故有$$\delta B_z^2 > \delta B_\theta^2 \rightarrow 2B_{z 0}^2 > B_{\theta 0}^2$$

对于$m=1$的弯曲不稳定性,只要收到扰动时,内部指向圆心的张力形成的向左合力大于外部向内的压力形成的向右合力为,就是稳定的。假设半径缩小$\delta r$,则

内部指向圆心的张力形成的向左合力:$$F_1 = \frac{B_z^2}{\mu_0} \cdot \frac{1}{R} \times \pi a^2 \lambda = \frac{B_z^2}{\mu_0} \pi a^2 \times 2\sin\theta$$

外部向内的压力形成的向右合力:$$F_2 = 2\sin\theta \int_a^\lambda \frac{B_\theta^2(r)}{2\mu_0} 2\pi r dr = 2\sin\theta \frac{B_\theta^2(a)}{\mu_0} \pi a^2 \ln\frac{\lambda}{a}$$

故有$$F_1 > F_2 \rightarrow B_z^2 > B_\theta^2 \ln\frac{\lambda}{a}$$

(3):

$$B_z^2 > B_\theta^2 \ln\frac{\lambda}{a}$$

\end{document}
