\documentclass{article}
\usepackage[UTF8]{ctex}  % 使用中文支持包
\usepackage[a4paper, margin=1in]{geometry}  % 设置纸张大小和边距
\usepackage{anyfontsize}  % 解决字体大小报错问题
\usepackage{fancyhdr}  % 设置页眉、页脚、页码
\usepackage{longtable}  % 支持长表格

\usepackage{amsmath}  % 数学公式支持
\usepackage{cases}  % 支持联立编号

\usepackage{graphicx}  % 插入图片支持
\usepackage{float}  % 设置图片浮动位置
\usepackage{subfigure}  % 插入多图时用子图显示

\usepackage{listings}  % 代码块支持
\usepackage{xcolor}  % 设置代码块颜色

\usepackage[hyphens]{url}  % 支持链接换行
\usepackage{hyperref}  % 超链接支持
\usepackage{lastpage}  % 添加lastpage包
\usepackage{gbt7714}  %国标参考文献
\bibliographystyle{gbt7714-numerical}
\hypersetup{
    hidelinks,
    colorlinks=true,
    allcolors=black,
    pdfstartview=Fit,
    breaklinks=true
}

\title{射线源导论-第十三周作业}
\author{\LaTeX\ by\ 何宇峰\ }
\date{\today}
\pagenumbering{arabic}

\begin{document}
\pagestyle{fancy}

\fancyhead[L]{何宇峰}
\fancyhead[C]{射线源导论-第十三周作业}
\fancyhead[R]{\today}
\fancyfoot[C]{Page \thepage/\pageref{LastPage}}

\section*{第十三周课程作业}

% 第一题
\subsection*{瓶方法}

将中子限制在一个物理瓶子或磁瓶中,并测量经过一段时间后剩余的中子数来确定中子的寿命。

\subsection*{束方法}

通过测量中子束中发生β衰变的事件数来确定中子的寿命。由于中子穿透力很强,因此如果中子泄露出容器,那么计算时会把这部分归于β衰变,导致中子测量寿命偏短。同时中子也可能和瓶壁发生反应。这方面的问题可以通过使用更大、更光滑的容器、更慢的中子,或者使用磁场来束缚中子来减少误差。除此之外中子可能通过一些未知的途径转变,测量寿命也可能偏长。

% 第二题
\section*{第二题}

简单估计反应堆中子源$MW$可以获得的热中子通量。

每次裂变沉积热200MeV,裂变反应产生的所有中子散步在慢化器中约20000cm²的球面上。(半径等于10~15cm)

对于1MW的反应堆,裂变数

$$ n = \frac{1M}{200MeV} = 3.125 \times 10^{17} $$

以U235计算,每次裂变产生中子数$\mu = 2.43$

故每秒产生中子

$$ n_{总} = n \times \mu = 7.594 \times 10^{10} $$

对应中子通量为

$$ \phi = \frac{n_{总}}{S} = 3.797 \times 10^{15}/(s \cdot cm^2) $$


\end{document}
