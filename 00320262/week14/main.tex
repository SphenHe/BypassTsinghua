\documentclass{article}
\usepackage[UTF8]{ctex}  % 使用中文支持包
\usepackage[a4paper, margin=1in]{geometry}  % 设置纸张大小和边距
\usepackage{anyfontsize}  % 解决字体大小报错问题
\usepackage{fancyhdr}  % 设置页眉、页脚、页码
\usepackage{longtable}  % 支持长表格

\usepackage{amsmath}  % 数学公式支持
\usepackage{cases}  % 支持联立编号

\usepackage{graphicx}  % 插入图片支持
\usepackage{float}  % 设置图片浮动位置
\usepackage{subfigure}  % 插入多图时用子图显示

\usepackage{listings}  % 代码块支持
\usepackage{xcolor}  % 设置代码块颜色

\usepackage[hyphens]{url}  % 支持链接换行
\usepackage{hyperref}  % 超链接支持
\usepackage{lastpage}  % 添加lastpage包
\usepackage{gbt7714}  %国标参考文献
\bibliographystyle{gbt7714-numerical}
\hypersetup{
    hidelinks,
    colorlinks=true,
    allcolors=black,
    pdfstartview=Fit,
    breaklinks=true
}

\title{射线源导论-第十三周作业}
\author{\LaTeX\ by\ 何宇峰\ }
\date{\today}
\pagenumbering{arabic}

\begin{document}
\pagestyle{fancy}

\fancyhead[L]{何宇峰}
\fancyhead[C]{射线源导论-第十三周作业}
\fancyhead[R]{\today}
\fancyfoot[C]{Page \thepage/\pageref{LastPage}}

\section*{第十三周课程作业}

\subsection*{如果要得到高电荷态的正离子源,你会选择哪种类型的离子源}

ECR 离子源。正离子被磁场限制多次电离,可以得到高电荷态的正离子源。

\subsection*{负离子产生的机制有哪些?}

表面、体积和电荷交换等产生过程。

\subsection*{通过表面效应产生负离子的流强得到提高,有何措施?}

利用铯效应。比如在磁控管源中添加铯蒸汽,还有一些其他依赖铯化表面的源比如表面转换器源,潘宁型负离子源。

此外,也可以通过偏心引出、设置铯炉的温度控制铯进入等离子体的通量等方式提高流强。

\subsection*{体产生负离子的机理?具体技术措施?}

高振动激发的氢分子通过解离电子附着过程捕获低能等离子体电子以形成负氢离子。

通过磁性过滤器,发生接力电子附着的等离子体区域应与进行氢分子激发的等离子体区域分开,需要相对高能量的电子加速器进行激发。

使用灯丝多会切离子源,在引出区域附近增加一个磁偶极子过滤场,阻止高能电子进入引出区域。而离子、分子、电子可以通过。这样放电就被有效分为两个区域,其中一个是氢负离子产生的低温等离子体

\subsection*{选择离子源的某个应用,如科学研究的加速器、离子注入机、离子光刻、肿瘤治疗等,了解所需要的离子电荷态、束流强度、束流能量、束流发射度、时间结构。
}

肿瘤治疗,以重离子治疗为例

\begin{itemize}
    \item 离子电荷态:高电荷态的离子,如 C$^{6+}$
    \item 束流强度:一般在 nA 至 μA 的范围内,典型值可能在 100-200 nA 左右。
    \item 束流能量:离子需要到达并穿透目标肿瘤位置,同时释放最大能量,到达布拉格峰。例如 200-400 MeV/u。
    \item 束流发射度:低,以便精确聚焦。例如在 0.1 $\pi$·mm·mrad 左右。
    \item 时间结构:脉冲。脉冲束流可以更精确地控制剂量分布,减少健康组织的暴露时间。
\end{itemize}

\end{document}
