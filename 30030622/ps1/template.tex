\documentclass[11pt]{article}

\usepackage[tbtags]{amsmath}
\usepackage{amssymb}
\usepackage{epsfig}
\usepackage{import}
\usepackage{mathrsfs}
\usepackage{spacing}

% GWW DEFINITIONS AND ABBREVIATIONS

% TeX Defs

\usepackage[tbtags]{amsmath} % defines many math commands and
			     % subequations environment, etc  
\usepackage{amssymb}  % get, among others, blackboard bold fonts
                      % defines extra symbols like \gtreqless, etc
\usepackage{verbatim} % get comment environment, + new verbatim 
% \usepackage{amsxtra}  % get, eg, \accentedsymbol

\DeclareMathOperator*{\argmax}{arg\,max}
\DeclareMathOperator*{\argmin}{arg\,min}
\DeclareMathOperator*{\argsup}{arg\,sup}
\DeclareMathOperator*{\arginf}{arg\,inf}
\DeclareMathOperator{\erfc}{erfc}
\DeclareMathOperator{\diag}{diag}
\DeclareMathOperator{\cum}{cum}
\DeclareMathOperator{\sgn}{sgn}
\DeclareMathOperator{\tr}{tr}
\DeclareMathOperator{\spn}{span}
\DeclareMathOperator{\adj}{adj}
\DeclareMathOperator{\var}{var}
\DeclareMathOperator{\cov}{cov}
\DeclareMathOperator{\sech}{sech}
\DeclareMathOperator{\sinc}{sinc}
\DeclareMathOperator*{\Dir}{Dir}
\DeclareMathOperator*{\lms}{l.i.m.\,}
\newcommand{\varop}[1]{\var\left[{#1}\right]}
\newcommand{\covop}[2]{\cov\left({#1},{#2}\right)}

\newcommand{\p}{\partial}

\newcommand\Perp{\protect\mathpalette{\protect\independenT}{\perp}}
\def\independenT#1#2{\mathrel{\rlap{$#1#2$}\mkern3mu{#1#2}}}

\newcommand\indep{\protect\mathpalette{\protect\independenT}{\perp}}
\def\independenT#1#2{\mathrel{\rlap{$#1#2$}\mkern5mu{#1#2}}}


% LIST ENVIRONMENTS

\newcounter{actr}
\newenvironment{alist}%
{\begin{list}{(\alph{actr})}{\usecounter{actr}}}{\end{list}}

\newcounter{ictr}
\newenvironment{ilist}%
{\begin{list}{(\roman{ictr})}{\usecounter{ictr}}}{\end{list}}

\iffalse

% SPACING ENVIRONMENTS 

\newenvironment{singlespace}%
{\begin{spacing}{1}}{\end{spacing}}

\newenvironment{onehalfspace}% for 11pt font
{\begin{spacing}{1.21}}{\end{spacing}}

\newenvironment{doublespace}% for 11pt font
{\begin{spacing}{1.62}}{\end{spacing}}

\fi

% THEOREM ENVIRONMENTS

\newtheorem{thm}{Theorem}
\newtheorem{lemma}{Lemma}
\newtheorem{claim}{Claim}
\newtheorem{corol}{Corollary}
\newtheorem{prop}{Proposition}
\newtheorem{defn}{Definition}
\newenvironment{proof}%
{\noindent{\em Proof: } \begin{singlespace} \small \noindent}%
{\noindent\qed \end{singlespace}}
\newenvironment{new-proof}[1]%
{{\em Proof of #1: } \begin{singlespace} \small \noindent}%
{\ \noindent\qed \end{singlespace}}

\newcommand{\abs}[1]{\left|#1\right|}
%\newcommand{\comb}[2]{{#1\choose#2}}
\newcommand{\comb}[2]{\binom{#1}{#2}}
\newcommand{\ie}{i.e.}
\newcommand{\eg}{e.g.}
\newcommand{\etc}{etc.}
\newcommand{\viz}{viz.}
\newcommand{\etal}{et al.}
\newcommand{\cf}{cf.}

\newcommand{\vect}[3]{\begin{bmatrix} #1 & #2 & \cdots & #3 \end{bmatrix}^\T}

\newcommand{\dsp}{.5\baselineskip}             % double space amount
\newcommand{\down}{\vspace{\dsp}}              % double space command
\newcommand{\ddown}{\vspace{\baselineskip}}    % quadruple space command
\newcommand{\spec}{\hspace*{1pt}}              % little bit of space
\newcommand{\ds}{\displaystyle}                % abbreviation
\newcommand{\ts}{\textstyle}                % abbreviation
\newcommand{\nin}{\noindent}                   % noindent abbreviation
\newcommand{\cvar}[1]{\mathrm{var_{#1}\,}}
\newcommand{\qed}{\rule[0.1ex]{1.4ex}{1.6ex}}
\newcommand{\mycap}[2]{\caption{\sl #2 \label{#1}}}
\newcommand{\subcap}[1]{{\begin{center}\sl #1\end{center}}}
\newcommand{\ditem}[1]{\item[#1 \hspace*{\fill}]}
\newcommand{\appfig}{\vspace*{1in}\begin{center} Figure appended to
                       end of manuscript. \end{center} \vspace*{1in}}
\newcommand{\psx}[1]{\centerline{\epsfxsize=6in \epsfbox{#1}}}
\newcommand{\psy}[1]{\centerline{\epsfysize=7in \epsfbox{#1}}}
\newcommand{\psxs}[2]{\centerline{\epsfxsize=#1in \epsfbox{#2}}}
\newcommand{\psxsbb}[3]{\centerline{\epsfxsize=#1in \epsfbox[#3]{#2}}}
\newcommand{\psys}[2]{\centerline{\epsfysize=#1in \epsfbox{#2}}}
\newcommand{\convsamp}[3]{\left.\left\{#1 \ast #2\right\}\right|_{#3}}
\newcommand{\gap}{\qquad}
\newcommand{\order}[1]{\mathcal{O}\left(#1\right)}
\newcommand{\arror}[3]{\begin{cases} #1 & #2 \\ 
                                     #3 & \text{otherwise} \end{cases}}
\newcommand{\arrorc}[3]{\begin{cases} #1 & #2 \\ 
                                     #3 & \text{otherwise,} \end{cases}}
\newcommand{\arrorp}[3]{\begin{cases} #1 & #2 \\ 
                                     #3 & \text{otherwise.} \end{cases}}
\newcommand{\darror}[4]{\begin{cases} #1 & #2 \\ #3 & #4 \end{cases}}
% \newcommand{\defeq}{\stackrel{\triangle}{=}}
\newcommand{\defeq}{\stackrel{\Delta}{=}}
\newcommand{\msconv}{\stackrel{\mathrm{m.s.}}{\longrightarrow}}
\newcommand{\pwaeconv}{\stackrel{\mathrm{p.w.a.e.}}{\longrightarrow}}
\newcommand{\peq}{\stackrel{\mathcal{P}}{=}}
% \newcommand{\glt}{ \begin{array}{c} \Hh=H_1 \\ 
%  \renewcommand{\arraystretch}{.3} 
%  \begin{array}{c} > \\ < \end{array}
%  \renewcommand{\arraystretch}{1} \\ \Hh=H_0 \end{array}}

\hyphenation{or-tho-nor-mal}
\hyphenation{wave-let wave-lets}

\newcommand{\crb}{Cram\'{e}r-Rao}  % obsolete
\newcommand{\CR}{Cram\'{e}r-Rao}
\newcommand{\KL}{Karhunen-Lo\`{e}ve}
\newcommand{\sE}{\sqrt{E_0}}
\newcommand{\pe}{\Pr(\eps)}
\newcommand{\jw}{j\w}
\newcommand{\ejw}{e^{j\w}}
\newcommand{\ejv}{e^{j\nu}}
\newcommand{\wo}{{\w_0}}
\newcommand{\woh}{{\wh_0}}
\newcommand{\sumi}[1]{\sum_{#1=-\infty}^{+\infty}}
\newcommand{\inti}{\int_{-\infty}^{+\infty}}
\newcommand{\intp}{\int_{-\pi}^{\pi}}
\newcommand{\nintp}{\frac{1}{2\pi}\int_{-\pi}^{\pi}}
\newcommand{\inth}{\int_{0}^{\infty}}
\newcommand{\ev}{\mathbb{E}}
\newcommand{\E}[1]{\ev\left[{#1}\right]}
\newcommand{\Ed}[2]{\ev_{#1}\left[{#2}\right]} % Expectation wrt to distr.
\newcommand{\bigE}[1]{E\bigl[{#1}\bigr]}
\newcommand{\BigE}[1]{E\Bigl[{#1}\Bigr]}
\newcommand{\biggE}[1]{E\biggl[{#1}\biggr]}
\newcommand{\BiggE}[1]{E\Biggl[{#1}\Biggr]}
\newcommand{\Prob}[1]{\Pr\left[{#1}\right]}
\newcommand{\Pu}[1]{\Pr\left[{#1}\right]} % obsolete; same as \Prob now
\newcommand{\Pc}[2]{\Pr\left[{#1}\mid{#2}\right]}  % obsolete
\newcommand{\Pcb}[2]{\Pr\left[{#1}\Bigm|{#2}\right]} % obsolete
\newcommand{\Q}[1]{\mathcal{Q}\left({#1}\right)}
\newcommand{\FT}[1]{\mathcal{F}\left\{{#1}\right\}}
\newcommand{\LT}[1]{\mathcal{L}\left\{{#1}\right\}}
\newcommand{\ZT}[1]{\mathcal{Z}\left\{{#1}\right\}}
%\newcommand{\reals}{\mathbf{R}}
\newcommand{\reals}{\mathbb{R}}
%\newcommand{\ints}{\mathbf{Z}}
\newcommand{\ints}{\mathbb{Z}}
\newcommand{\compls}{\mathbb{C}}
\newcommand{\nats}{\mathbb{N}}
\newcommand{\rats}{\mathbb{Q}}
\newcommand{\ltwor}{L^2(\reals)}
\newcommand{\ltwoz}{\ell^2(\ints)}
\newcommand{\ltwow}{L^2(\Omega)}
% \newcommand{\ltwo}{\mathbf{L}^2}
% \newcommand{\ltwor}{\mathbf{L}^2 (\reals)}
% \newcommand{\ltwoz}{\mathbf{l}^2 (\ints)}
\newcommand{\sys}[1]{\mathcal{S}\left\{#1\right\}}
\newcommand{\nn}{\nonumber}

\newcommand{\ip}[2]{\left\langle{#1},{#2}\right\rangle}
\newcommand{\di}[2]{d\left({#1},{#2}\right)}
\newcommand{\ceil}[1]{\lceil{#1}\rceil}
\newcommand{\floor}[1]{\lfloor{#1}\rfloor}
\newcommand{\phase}{\measuredangle}

\newcommand{\Ht}{\mathrm{H}}
\newcommand{\T}{{\mathrm{T}}}
% \newcommand{\R}{\Re\mathit{e}}
% \newcommand{\I}{\Im\mathit{m}}
\DeclareMathOperator{\R}{Re}
\DeclareMathOperator{\I}{Im}

\newcommand{\theterm}{Spring 2025}

\newcommand{\thecoursename}{
Tsinghua University \\
Department of Civil Engineering \\
\vspace*{0.1in}
\textsc{Mathematical Modeling and Data Analysis}
}


\newcommand{\courseheader}{
\vspace*{-1in}
\begin{center}
\thecoursename \\
\theterm
\vspace*{0.1in}
\hrule
\end{center}
}


\newcounter{psctr}
%\newcounter{probctr}[psctr]
%\renewcommand{\theprobctr}{\arabic{psctr}.\arabic{probctr}}
\newcounter{probctr}
\newcommand{\problem}[1]{%
\addtocounter{probctr}{1}
\vspace{.15in}

%\noindent\textbf{Problem \thepsctr.\theprobctr}\nopagebreak
\noindent\textbf{Problem \theprobctr}\nopagebreak
\noindent{#1}

}
\newcommand{\extraproblem}[1]{%
\addtocounter{probctr}{1}
\vspace{.15in}

\noindent\textbf{Problem \thepsctr.\theprobctr\ (Practice)}\nopagebreak

\noindent{#1}

}

\newcommand{\pgmproblem}[2]{%
\addtocounter{probctr}{1}
\vspace{.15in}

\noindent\textbf{Problem \thepsctr.\theprobctr\ (Exercise #1 in Koller/Friedman)}\nopagebreak

\noindent{#2}

}

\newcommand{\pgmextraproblem}[2]{%
\addtocounter{probctr}{1}
\vspace{.15in}

\noindent\textbf{Problem \thepsctr.\theprobctr\ (Exercise #1 in Koller/Friedman) (Practice)}\nopagebreak

\noindent{#2}

}

\DeclareMathAlphabet{\mathbsf}{OT1}{cmss}{bx}{n}% bold sans serif
\DeclareMathAlphabet{\mathssf}{OT1}{cmss}{m}{sl}% slanted sans serif
\DeclareMathAlphabet{\mathbbb}{U}{bbold}{m}{n}  % bb bold numbers


% define some useful uppercase Greek letters in regular and bold sf
\DeclareSymbolFont{bsfletters}{OT1}{cmss}{bx}{n}  
\DeclareSymbolFont{ssfletters}{OT1}{cmss}{m}{n}
\DeclareMathSymbol{\bsfGamma}{0}{bsfletters}{'000}
\DeclareMathSymbol{\ssfGamma}{0}{ssfletters}{'000}
\DeclareMathSymbol{\bsfDelta}{0}{bsfletters}{'001}
\DeclareMathSymbol{\ssfDelta}{0}{ssfletters}{'001}
\DeclareMathSymbol{\bsfTheta}{0}{bsfletters}{'002}
\DeclareMathSymbol{\ssfTheta}{0}{ssfletters}{'002}
\DeclareMathSymbol{\bsfLambda}{0}{bsfletters}{'003}
\DeclareMathSymbol{\ssfLambda}{0}{ssfletters}{'003}
\DeclareMathSymbol{\bsfXi}{0}{bsfletters}{'004}
\DeclareMathSymbol{\ssfXi}{0}{ssfletters}{'004}
\DeclareMathSymbol{\bsfPi}{0}{bsfletters}{'005}
\DeclareMathSymbol{\ssfPi}{0}{ssfletters}{'005}
\DeclareMathSymbol{\bsfSigma}{0}{bsfletters}{'006}
\DeclareMathSymbol{\ssfSigma}{0}{ssfletters}{'006}
\DeclareMathSymbol{\bsfUpsilon}{0}{bsfletters}{'007}
\DeclareMathSymbol{\ssfUpsilon}{0}{ssfletters}{'007}
\DeclareMathSymbol{\bsfPhi}{0}{bsfletters}{'010}
\DeclareMathSymbol{\ssfPhi}{0}{ssfletters}{'010}
\DeclareMathSymbol{\bsfPsi}{0}{bsfletters}{'011}
\DeclareMathSymbol{\ssfPsi}{0}{ssfletters}{'011}
\DeclareMathSymbol{\bsfOmega}{0}{bsfletters}{'012}
\DeclareMathSymbol{\ssfOmega}{0}{ssfletters}{'012}

\newcommand{\fxfm}{\stackrel{\mathcal{F}}{\longleftrightarrow}}
\newcommand{\lxfm}{\stackrel{\mathcal{L}}{\longleftrightarrow}}
\newcommand{\zxfm}{\stackrel{\mathcal{Z}}{\longleftrightarrow}}

\DeclareMathOperator*{\gltop}{\gtreqless}
\newcommand{\glt}{\;\gltop^{\Hh=\svH_1}_{\Hh=\svH_0}\;}
\newcommand{\glty}{\;\gltop^{\Hh(\svy)=\svH_1}_{\Hh(\svy)=\svH_0}\;}
\newcommand{\gltby}{\;\gltop^{\Hh(\svby)=\svH_1}_{\Hh(\svby)=\svH_0}\;}
\DeclareMathOperator*{\geltop}{\genfrac{}{}{0pt}{}{\ge}{<}}
\newcommand{\gelty}{\;\geltop^{\Hh(\svy)=\svH_1}_{\Hh(\svy)=\svH_0}\;}
\newcommand{\geltby}{\;\geltop^{\Hh(\svby)=\svH_1}_{\Hh(\svby)=\svH_0}\;}
\renewcommand{\pe}{\Pr(e)}
\renewcommand{\defeq}{\triangleq}
\newcommand{\like}{\svlike}
\newcommand{\rvlike}{\mathssf{L}}
\newcommand{\sst}{\cl}
\newcommand{\svlike}{L}
\newcommand{\llike}{\rvllike}
\newcommand{\rvllike}{\cl}
\newcommand{\svllike}{l}
\newcommand{\bllike}{\rvbllike}
\newcommand{\rvbllike}{\boldsymbol{\cl}}
\newcommand{\svbllike}{\mathbf{l}}
\newcommand{\Qb}{\overline{Q}}
\renewcommand{\comb}[2]{\binom{#1}{#2}}


%% Random/sample variable/vector declarations.  Please add in alphabetical
%% order.  First section is for capitals.  Second for lower case.
% Capitals
\newcommand{\rvA}{{\mathssf{A}}}	% A
\newcommand{\svA}{A}
\newcommand{\rvbA}{{\mathbsf{A}}}
\newcommand{\svbA}{{\mathbf{A}}}
\newcommand{\rvFh}{{\hat{\mathssf{F}}}}	% F
\newcommand{\rvF}{{\mathssf{F}}}
\newcommand{\rvHh}{{\hat{\mathssf{H}}}}	% H
\newcommand{\rvH}{{\mathssf{H}}}
\newcommand{\svH}{H}
\newcommand{\svHh}{{\hat{\svH}}}
\newcommand{\rvL}{{\mathssf{L}}}	% L
\newcommand{\rvN}{{\mathssf{N}}}	% N
\newcommand{\rvR}{{\mathssf{R}}}	% R
\newcommand{\rvRh}{{\hat{\rvR}}}
\newcommand{\rvS}{{\mathssf{S}}}	% S
\newcommand{\rvSh}{{\hat{\rvS}}}
\newcommand{\rvW}{{\mathssf{W}}}	% W
\newcommand{\rvX}{{\mathssf{X}}}  	% X, random variable
\newcommand{\svX}{X}		  	
\newcommand{\rvXt}{{\tilde{\rvX}}}
\newcommand{\rvY}{{\mathssf{Y}}}	% Y
\newcommand{\rvZ}{{\mathssf{Z}}}	% Z

\newcommand{\rva}{{\mathssf{a}}}	% a
\newcommand{\rvah}{{\hat{\rva}}}
\newcommand{\sva}{a}
\newcommand{\svah}{{\hat{\sva}}}
\newcommand{\rvba}{{\mathbsf{a}}}
\newcommand{\svba}{{\mathbf{a}}}
\newcommand{\rvb}{{\mathssf{b}}}	% b
\newcommand{\rvc}{{\mathssf{c}}}	% c
\newcommand{\rvd}{{\mathssf{d}}}	% d
\newcommand{\rvbd}{{\mathbsf{d}}}	% d
\newcommand{\rve}{{\mathssf{e}}}	% e
\newcommand{\sve}{e}
\newcommand{\rvbfe}{{\mathbsf{e}}}
\newcommand{\svbfe}{{\mathbf{e}}}
\newcommand{\rvf}{{\mathssf{f}}}	% f
\newcommand{\svf}{f}
\newcommand{\rvbff}{{\mathbsf{f}}}
\newcommand{\svbff}{{\mathbf{f}}}
\newcommand{\rvh}{{\mathssf{h}}}	% h
\newcommand{\rvk}{{\mathssf{k}}}	% k
\newcommand{\svk}{k}
\newcommand{\rvm}{{\mathssf{m}}}	% m
\newcommand{\rvn}{{\mathssf{n}}}	% n
\newcommand{\rvbn}{{\mathbsf{n}}}
\newcommand{\rvq}{{\mathssf{q}}}	% q
\newcommand{\svq}{q}
\newcommand{\rvr}{{\mathssf{r}}}	% r
\newcommand{\svr}{r}
\newcommand{\rvs}{{\mathssf{s}}}	% s
\newcommand{\svs}{s}
\newcommand{\rvt}{{\mathssf{t}}}	% t
\newcommand{\svt}{t}
\newcommand{\rvu}{{\mathssf{u}}}	% u
\newcommand{\svu}{u}
\newcommand{\svuh}{{\hat{\svu}}}
\newcommand{\rvbu}{{\mathbsf{u}}}
\newcommand{\svbu}{{\mathbf{u}}}
\newcommand{\rvv}{{\mathssf{v}}}	% v
\newcommand{\svv}{v}
\newcommand{\svvh}{{\hat{\svv}}}
\newcommand{\rvbv}{{\mathbsf{v}}}
\newcommand{\svbv}{{\mathbf{v}}}
\newcommand{\rvvh}{{\hat{\rvv}}}
\newcommand{\rvw}{{\mathssf{w}}}	% w
\newcommand{\svw}{w}
\newcommand{\rvwh}{{\hat{\rvw}}}
\newcommand{\svwh}{{\hat{\svw}}}
\newcommand{\rvbw}{{\mathbsf{w}}}
\newcommand{\svbw}{{\mathbf{w}}}
\newcommand{\rvx}{{\mathssf{x}}}	% x, random variable
\newcommand{\rvxh}{{\hat{\rvx}}}
\newcommand{\rvxt}{{\tilde{\rvx}}}
\newcommand{\svx}{x}			% sample value
\newcommand{\svxh}{{\hat{\svx}}}
\newcommand{\svxt}{{\tilde{\svx}}}
\newcommand{\rvbx}{{\mathbsf{x}}}
\newcommand{\rvbxh}{{\hat{\rvbx}}}
\newcommand{\rvbxt}{{\tilde{\rvbx}}}
\newcommand{\svbx}{{\mathbf{\svx}}}
\newcommand{\svbxt}{{\tilde{\svbx}}}
\newcommand{\svbxh}{{\hat{\mathbf{x}}}}
\newcommand{\rvy}{{\mathssf{y}}}	% y
\newcommand{\rvyh}{{\hat{\mathssf{y}}}}
\newcommand{\svy}{y}
\newcommand{\rvyt}{{\tilde{\rvy}}}
\newcommand{\svyt}{{\tilde{\svy}}}
\newcommand{\svyh}{{\hat{\svy}}}
\newcommand{\rvby}{{\mathbsf{y}}}
\newcommand{\rvbyt}{{\tilde{\rvby}}}
\newcommand{\svby}{{\mathbf{y}}}
\newcommand{\svbyt}{{\tilde{\svby}}}
\newcommand{\rvz}{{\mathssf{z}}}	% z
\newcommand{\rvzh}{{\hat{\rvz}}}
\newcommand{\rvzt}{{\tilde{\rvz}}}
\newcommand{\svz}{z}
\newcommand{\svzh}{{\hat{\svz}}}
\newcommand{\rvbz}{{\mathbsf{z}}}
\newcommand{\svbz}{{\mathbf{z}}}

% vectors

\newcommand{\bzero}{{\mathbf{0}}}
\newcommand{\bOne}{{\mathbf{1}}}

% Handle uppercase Greek differently
\newcommand{\rvTh}{\ssfTheta}
\newcommand{\svTh}{\Theta}
\newcommand{\rvbTh}{\bsfTheta}
\newcommand{\svbTh}{\boldsymbol{\Theta}}
\newcommand{\rvPh}{\ssfPhi}
\newcommand{\svPh}{\Phi}
\newcommand{\rvbPh}{\bsfPhi}
\newcommand{\svbPh}{\boldsymbol{\Phi}}

\newcommand{\ddx}{\frac{\p}{\p \svx}}
\newcommand{\ddbx}{\frac{\p}{\p\svbx}}

%  --add new macros below this line--
\newcommand{\ybar}{\ensuremath{\bar{y}}}
\newcommand{\xbar}{\ensuremath{\bar{x}}}

\newcommand{\gauss}{\mathtt{N}}
\newcommand{\bern}{\mathtt{B}}
\newcommand{\unif}{\mathtt{U}}
\newcommand{\poiss}{\mathtt{P}}
\newcommand{\expfam}{\mathtt{E}}

\newcommand{\kron}{\mathbbb{1}}


\setcounter{psctr}{1} % Pset counter

\begin{document}
\courseheader

\begin{center}
  \underline{\bf Problem Set \thepsctr}
\end{center}

\problem{ }
% Once upon a time at Tsinghua University in Beijing...

Given Special matrices

\begin{equation*}
  K_n =
  \left[
  \begin{matrix}
    2 & -1 & 0 & 0 & \cdots & 0 \\
    -1 & 2 & -1 & 0 & \cdots & 0 \\
    0 & -1 & 2 & -1 & \cdots & 0 \\
    \ldots & \ldots & \ldots & \ldots & \ldots & \ldots \\
    0 & 0 & \cdots & -1 & 2 & -1 \\
    0 & 0 & \cdots & 0 & -1 & 2
  \end{matrix}
  \right]
\end{equation*}

(a) Compute the $K_nv_i$ for $i = 1, 2, 3$

$v_1 = [\sin t, \cdots, \sin nt]^{T}$

\begin{equation*}
  K_nv_1 =
  \left[
  \begin{matrix}
    2 \sin t - \sin 2t \\
    -\sin t + 2 \sin 2t - \sin 3t \\
    -\sin 2t + 2 \sin 3t - \sin 4t \\
    \cdots \\
    -\sin (n-1)t + 2 \sin nt
  \end{matrix}
  \right]
  =
  \left[
  \begin{matrix}
    2 \sin t (1 - \cos t) \\
    \cdots \\
    2 \sin(jt) (1 - \cos t) \\
    \cdots \\
    -\sin (n-1)t (1 - \cos t)
  \end{matrix}
  \right]
\end{equation*}

$v_2 = [\cos t, \cdots, \cos nt]^{T}$

\begin{equation*}
  K_nv_2 =
  \left[
  \begin{matrix}
    2 \cos t - \cos 2t \\
    -\cos t + 2 \cos 2t - \cos 3t \\
    -\cos 2t + 2 \cos 3t - \cos 4t \\
    \cdots \\
    -\cos (n-1)t + 2 \cos nt
  \end{matrix}
  \right]
  =
  \left[
  \begin{matrix}
    2 \cos t (1 - \cos t) \\
    \cdots\\
    2 \cos(jt) (1 - \cos t) \\
    \cdots \\
    -\cos (n-1)t (1 - \cos t)
  \end{matrix}
  \right]
\end{equation*}

$v_3 = [e^{it}, \cdots, e^{int}]^{T}$

\begin{equation*}
  K_nv_3 =
  \left[
  \begin{matrix}
    2 e^{it} - e^{2it} \\
    -e^{it} + 2 e^{2it} - e^{3it} \\
    -e^{2it} + 2 e^{3it} - e^{4it} \\
    \cdots \\
    -e^{(n-1)it} + 2 e^{int}
  \end{matrix}
  \right]
  =
  \left[
  \begin{matrix}
    2 e^{it} (1 - \cos t) \\
    \cdots \\
    2 e^{jit} (1 - \cos t) \\
    \cdots \\
    2e^{(n-1)it} (1 - \cos t)
  \end{matrix}
  \right]
\end{equation*}

(b) Determine the eigenvectors of $K_n$

Guess the eigen value is $(2-2\cos t)$. Because we can see $K_nv_1 = (2-2\cos t)v_1$, $K_nv_2 = (2-2\cos t)v_2$, $K_nv_3 = (2-2\cos t)v_3$.

Then we can see the eigenvectors are $v_1, v_2, v_3$.

(c) Define a modified matrix $M_n$ by replacing the diagonal entries in $K_n$ with $\sqrt{2}$:

We can find that $M_3$ is not invertible so we can see that $M_n$ is not invertible.

The eigenvectors of $M_n$ are the same as the eigenvectors of $K_n$.(I guess)

\problem{ }

Define a permuation matrix $P_4$ as follows:

\begin{equation*}
  P_4 =
  \left[
  \begin{matrix}
    0 & 1 & 0 & 0 \\
    0 & 0 & 1 & 0 \\
    0 & 0 & 0 & 1 \\
    1 & 0 & 0 & 0
  \end{matrix}
  \right]
\end{equation*}

(a) Factor $P_4$ as $P_4 = F_4\Gamma_4F_4^{-1}$

\begin{equation*}
  P_4 - \lambda =
  \left[
  \begin{matrix}
    -\lambda & 1 & 0 & 0 \\
    0 & -\lambda & 1 & 0 \\
    0 & 0 & -\lambda & 1 \\
    1 & 0 & 0 & -\lambda
  \end{matrix}
  \right]
  =
  0
\end{equation*}

Solve the equation and we get the eigenvalues are $\lambda = 1, -1, i, -i$, and the eigenvectors are $v_1 = [1, 1, 1, 1]^{T}$, $v_2 = [1, -1, 1, -1]^{T}$, $v_3 = [1, -i, -1, i]^{T}$, $v_4 = [1, i, -1, -i]^{T}$.
Then we can get the matrix $F_4$ and $\Gamma_4$.

\begin{equation*}
  F_4 =
  \left[
  \begin{matrix}
    1 & 1 & 1 & 1 \\
    1 & -1 & -i & i \\
    1 & 1 & -1 & -1 \\
    1 & -1 & i & -i
  \end{matrix}
  \right]
\end{equation*}

\begin{equation*}
  \Gamma_4 =
  \left[
  \begin{matrix}
    1 & 0 & 0 & 0 \\
    0 & -1 & 0 & 0 \\
    0 & 0 & i & 0 \\
    0 & 0 & 0 & -i
  \end{matrix}
  \right]
\end{equation*}

(b) Prove that $C_4 = 2I -P_4 - P_4^3$

\begin{equation*}
  \begin{aligned}
    P_4^3 & = F_4\Gamma_4^3F_4^{-1} \\
    & =
    \left[
      \begin{matrix}
        1 & 1 & 1 & 1 \\
        1 & -1 & -i & i \\
        1 & 1 & -1 & -1 \\
        1 & -1 & i & -i
      \end{matrix}
    \right]
    \left[
    \begin{matrix}
      1 & 0 & 0 & 0 \\
      0 & -1 & 0 & 0 \\
      0 & 0 & i & 0 \\
      0 & 0 & 0 & -i
    \end{matrix}
    \right]^3
    \left[
      \begin{matrix}
        1 & 1 & 1 & 1 \\
        1 & -1 & -i & i \\
        1 & 1 & -1 & -1 \\
        1 & -1 & i & -i
      \end{matrix}
    \right]^{-1} \\
    & =
    \left[
      \begin{matrix}
        1 & 1 & 1 & 1 \\
        1 & -1 & -i & i \\
        1 & 1 & -1 & -1 \\
        1 & -1 & i & -i
      \end{matrix}
    \right]
    \left[
    \begin{matrix}
      1 & 0 & 0 & 0 \\
      0 & -1 & 0 & 0 \\
      0 & 0 & -i & 0 \\
      0 & 0 & 0 & i
    \end{matrix}
    \right]
    \left[
    \begin{matrix}
      1 & 1 & 1 & 1 \\
      1 & -1 & -i & i \\
      1 & 1 & -1 & -1 \\
      1 & -1 & i & -i
    \end{matrix}
    \right]^{-1} \\
    & =
    \left[
    \begin{matrix}
      0 & 0 & 0 & 1 \\
      1 & 0 & 0 & 0 \\
      0 & 1 & 0 & 0 \\
      0 & 0 & 1 & 0
    \end{matrix}
    \right]
  \end{aligned}
\end{equation*}

Then

\begin{equation*}
  \begin{aligned}
    P_4 + P_4^3 & =
    \left[
    \begin{matrix}
      0 & 0 & 0 & 1 \\
      1 & 0 & 0 & 0 \\
      0 & 1 & 0 & 0 \\
      0 & 0 & 1 & 0
    \end{matrix}
    \right]
    +
    \left[
    \begin{matrix}
      0 & 1 & 0 & 0 \\
      0 & 0 & 1 & 0 \\
      0 & 0 & 0 & 1 \\
      1 & 0 & 0 & 0
    \end{matrix}
    \right] \\
    & =
    \left[
    \begin{matrix}
      0 & 1 & 0 & 1 \\
      1 & 0 & 1 & 0 \\
      0 & 1 & 0 & 1 \\
      1 & 0 & 1 & 0
    \end{matrix}
    \right]
  \end{aligned}
\end{equation*}

We know that

\begin{equation*}
  C_4 =
  \left[
  \begin{matrix}
    2 & -1 & 0 & -1 \\
    -1 & 2 & -1 & 0 \\
    0 & -1 & 2 & -1 \\
    -1 & 0 & -1 & 2
  \end{matrix}
  \right]
\end{equation*}

So we can see that $C_4 = 2I - P_4 - P_4^3$.

(c) Find the eigenvalues and eigenvectors of $C$ using (a) and (b)

We can see that $C_4 = 2I - P_4 - P_4^3 = 2I - F_4\Gamma_4F_4^{-1} - F_4\Gamma_4^3F_4^{-1} = F_4(2I - \Gamma_4 - \Gamma_4^3)F_4^{-1}$.

So the eigenvalues of $C_4$ are

\begin{equation*}
  \begin{aligned}
    2 - 1 - 1^3 & = 0 \\
    2 - (-1) - (-1)^3 & = 4 \\
    2 - i - i^3 & = 2 \\
    2 - (-i) - (-i)^3 & = 2
  \end{aligned}
\end{equation*}

The eigenvectors are

\begin{equation*}
  \begin{aligned}
    v_1 & = [1, 1, 1, 1]^{T} \\
    v_2 & = [1, -1, 1, -1]^{T} \\
    v_3 & = [0, 1, 0, -1]^{T} \\
    v_4 & = [1, 0, -1, 0]^{T}
  \end{aligned}
\end{equation*}

Optional:Given c = [c0 , c1 , c2, c3 ], how can you quickly compute the four components of F4 c if you know c0 + c2, c0 - c2 , c1 + c3 , c1 - c3 ?

\begin{equation*}
  F_4 c =
  \left[
    \begin{matrix}
      c_0+c_1+c_2+c_3 \\
      c_0+ic_1+i^2c_2+i^3c_3 \\
      c_0+i^2c_1+i^4c_2+i^6c_3 \\
      c_0+i^3c_1+i^6c_2+i^9c_3
    \end{matrix}
  \right]
\end{equation*}

We can see that

\begin{equation*}
  \begin{aligned}
    F_4 c_0 & = c_0 + c_1 + c_2 + c_3 \\
    F_4 c_1 & = c_0 - c_1 + ic_2 - ic_3 \\
    F_4 c_2 & = c_0 + c_1 - c_2 - c_3 \\
    F_4 c_3 & = c_0 - c_1 - ic_2 + ic_3
  \end{aligned}
\end{equation*}

Then we can quickly compute the four components of $F_4 c$ if we know $c_0 + c_2, c_0 - c_2, c_1 + c_3, c_1 - c_3$.

\problem{ }

Define a sum matrixas an upper triangular matrix of ones:

\begin{equation*}
  S_n =
  \left[
  \begin{matrix}
    1 & 1 & 1 & \cdots & 1 \\
    0 & 1 & 1 & \cdots & 1 \\
    0 & 0 & 1 & \cdots & 1 \\
    \cdots & \cdots & \cdots & \cdots & \cdots \\
    0 & 0 & 0 & \cdots & 1
  \end{matrix}
  \right]
\end{equation*}

(a) Find the inverse of $S_n$. Explain why the inverse of $S_n$ is a first differernce matrix.

We can see that $S_n$ is a upper triangular matrix, so the inverse of $S_n$ is also a upper triangular matrix.

\begin{equation*}
  S_n^{-1} =
  \left[
  \begin{matrix}
    1 & -1 & 0 & \cdots & 0 \\
    0 & 1 & -1 & \cdots & 0 \\
    0 & 0 & 1 & \cdots & 0 \\
    \cdots & \cdots & \cdots & \cdots & \cdots \\
    0 & 0 & 0 & \cdots & 1
  \end{matrix}
  \right]
\end{equation*}

So we can see that the inverse of $S_n$ is a first difference matrix.

When you multiply $S_n^{-1}$ with a vector, you can see that the result is the first difference of the vector. So the inverse of $S_n$ is a first difference matrix.

(b) Prove that $T_n^{-1}=S_nS_n^{T}$

We can see that

\begin{equation*}
  \begin{aligned}
    T_n & =
    \left[
    \begin{matrix}
      1 & -1 & 0 & \cdots & 0 \\
      -1 & 2 & -1 & \cdots & 0 \\
      0 & -1 & 2 & \cdots & 0 \\
      \cdots & \cdots & \cdots & \cdots & \cdots \\
      0 & 0 & 0 & \cdots & -1 \\
      0 & 0 & 0 & \cdots & 2
    \end{matrix}
    \right] \\
    & = LU \\
    & = (LU)^T \\
    & = U^TL^T \\
    & =
    \left[
    \begin{matrix}
      1 & -1 & 0 & \cdots & 0 \\
      0 & 1 & -1 & \cdots & 0 \\
      0 & 0 & 1 & \cdots & 0 \\
      \cdots & \cdots & \cdots & \cdots & \cdots \\
      0 & 0 & 0 & \cdots & 1
    \end{matrix}
    \right]^T
    \left[
      \begin{matrix}
        1 & 0 & 0 & \cdots & 0 \\
        -1 & 1 & 0 & \cdots & 0 \\
        0 & -1 & 1 & \cdots & 0 \\
        \cdots & \cdots & \cdots & \cdots & \cdots \\
        0 & 0 & 0 & \cdots & 1
      \end{matrix}
    \right]^T \\
    & = (S_n^T)^{-1}S_n^{-1} \\
    & = (S_nS_n^T)^{-1}
  \end{aligned}
\end{equation*}

Then we can see that $T_n^{-1} = S_nS_n^T$.



\problem{ }

(a) Find matrix $\Delta_{\_}$ that satisfies $K_{n-1} = \Delta_{\_}^T\Delta_{\_}$ and $B_{n} = \Delta_{\_}\Delta_{\_}^T$(Hint: $\Delta_{\_}$ is a backward difference matrix.)

\begin{equation*}
  K_{n-1} =
  \left[
  \begin{matrix}
    2 & -1 & 0 & 0 & \cdots & 0 \\
    -1 & 2 & -1 & 0 & \cdots & 0 \\
    0 & -1 & 2 & -1 & \cdots & 0 \\
    \ldots & \ldots & \ldots & \ldots & \ldots & \ldots \\
    0 & 0 & \cdots & -1 & 2 & -1 \\
    0 & 0 & \cdots & 0 & -1 & 2
  \end{matrix}
  \right]
\end{equation*}

\begin{equation*}
  B_n =
  \left[
  \begin{matrix}
    1 & -1 & 0 & 0 & \cdots & 0 \\
    -1 & 2 & -1 & 0 & \cdots & 0 \\
    0 & -1 & 2 & -1 & \cdots & 0 \\
    \ldots & \ldots & \ldots & \ldots & \ldots & \ldots \\
    0 & 0 & \cdots & -1 & 2 & -1 \\
    0 & 0 & \cdots & 0 & -1 & 1
  \end{matrix}
  \right]
\end{equation*}

So guess that $\Delta_{\_}$ is a backward difference matrix.

\begin{equation*}
  \Delta_{\_} =
  \left[
  \begin{matrix}
    1 & 0 & 0 & 0 & \cdots & 0 \\
    -1 & 1 & 0 & 0 & \cdots & 0 \\
    0 & -1 & 1 & 0 & \cdots & 0 \\
    \ldots & \ldots & \ldots & \ldots & \ldots & \ldots \\
    0 & 0 & \cdots & 0 & -1 & 1 \\
    0 & 0 & \cdots & 0 & 0 & -1
  \end{matrix}
  \right]
\end{equation*}

Then we can see that $\Delta_{\_}^T\Delta_{\_} = K_{n-1}$ and $\Delta_{\_}\Delta_{\_}^T = B_n$.

(b) Prove that $B_n$ is singular

Then we can see that

\begin{equation*}
  \begin{aligned}
    B_n &= LU \\
    &=
    \left[
    \begin{matrix}
      1 & 0 & 0 & 0 & \cdots & 0 \\
      -1 & 1 & 0 & 0 & \cdots & 0 \\
      0 & -1 & 1 & 0 & \cdots & 0 \\
      \ldots & \ldots & \ldots & \ldots & \ldots & \ldots \\
      0 & 0 & \cdots & -1 & 1 & 0 \\
      0 & 0 & \cdots & 0 & -1 & 1
    \end{matrix}
    \right]
    \left[
    \begin{matrix}
      1 & -1 & 0 & 0 & \cdots & 0 \\
      0 & 1 & -1 & 0 & \cdots & 0 \\
      0 & 0 & 1 & -1 & \cdots & 0 \\
      \ldots & \ldots & \ldots & \ldots & \ldots & \ldots \\
      0 & 0 & \cdots & 0 & 1 & -1 \\
      0 & 0 & \cdots & 0 & 0 & 0
    \end{matrix}
    \right] \\
  \end{aligned}
\end{equation*}

The $U$ matrix is not a full rank matrix, so the $B_n$ is singular.

(c) What condition should $f$ satisfy for $B_n v = f$ to be solvable?

We can see that $B_n$ is singular, so the $f$ should satisfy that $f$ is in the range of $B_n$. So $f$ should be in the range of $\Delta_{\_}$.

So rank of $f$ should be $n-1$.

So $f$ should satisfy that $\sum_{i=1}^{n} f_i = 0$.

\end{document}
