\newcommand{\theterm}{Spring 2025}

\newcommand{\thecoursename}{
Tsinghua University \\
Department of Civil Engineering \\
\vspace*{0.1in}
\textsc{Mathematical Modeling and Data Analysis}
}


\newcommand{\courseheader}{
\vspace*{-1in}
\begin{center}
\thecoursename \\
\theterm
\vspace*{0.1in}
\hrule
\end{center}
}


\newcounter{psctr}
%\newcounter{probctr}[psctr]
%\renewcommand{\theprobctr}{\arabic{psctr}.\arabic{probctr}}
\newcounter{probctr}
\newcommand{\problem}[1]{%
\addtocounter{probctr}{1}
\vspace{.15in}

%\noindent\textbf{Problem \thepsctr.\theprobctr}\nopagebreak
\noindent\textbf{Problem \theprobctr}\nopagebreak
\noindent{#1}

}
\newcommand{\extraproblem}[1]{%
\addtocounter{probctr}{1}
\vspace{.15in}

\noindent\textbf{Problem \thepsctr.\theprobctr\ (Practice)}\nopagebreak

\noindent{#1}

}

\newcommand{\pgmproblem}[2]{%
\addtocounter{probctr}{1}
\vspace{.15in}

\noindent\textbf{Problem \thepsctr.\theprobctr\ (Exercise #1 in Koller/Friedman)}\nopagebreak

\noindent{#2}

}

\newcommand{\pgmextraproblem}[2]{%
\addtocounter{probctr}{1}
\vspace{.15in}

\noindent\textbf{Problem \thepsctr.\theprobctr\ (Exercise #1 in Koller/Friedman) (Practice)}\nopagebreak

\noindent{#2}

}

\DeclareMathAlphabet{\mathbsf}{OT1}{cmss}{bx}{n}% bold sans serif
\DeclareMathAlphabet{\mathssf}{OT1}{cmss}{m}{sl}% slanted sans serif
\DeclareMathAlphabet{\mathbbb}{U}{bbold}{m}{n}  % bb bold numbers


% define some useful uppercase Greek letters in regular and bold sf
\DeclareSymbolFont{bsfletters}{OT1}{cmss}{bx}{n}  
\DeclareSymbolFont{ssfletters}{OT1}{cmss}{m}{n}
\DeclareMathSymbol{\bsfGamma}{0}{bsfletters}{'000}
\DeclareMathSymbol{\ssfGamma}{0}{ssfletters}{'000}
\DeclareMathSymbol{\bsfDelta}{0}{bsfletters}{'001}
\DeclareMathSymbol{\ssfDelta}{0}{ssfletters}{'001}
\DeclareMathSymbol{\bsfTheta}{0}{bsfletters}{'002}
\DeclareMathSymbol{\ssfTheta}{0}{ssfletters}{'002}
\DeclareMathSymbol{\bsfLambda}{0}{bsfletters}{'003}
\DeclareMathSymbol{\ssfLambda}{0}{ssfletters}{'003}
\DeclareMathSymbol{\bsfXi}{0}{bsfletters}{'004}
\DeclareMathSymbol{\ssfXi}{0}{ssfletters}{'004}
\DeclareMathSymbol{\bsfPi}{0}{bsfletters}{'005}
\DeclareMathSymbol{\ssfPi}{0}{ssfletters}{'005}
\DeclareMathSymbol{\bsfSigma}{0}{bsfletters}{'006}
\DeclareMathSymbol{\ssfSigma}{0}{ssfletters}{'006}
\DeclareMathSymbol{\bsfUpsilon}{0}{bsfletters}{'007}
\DeclareMathSymbol{\ssfUpsilon}{0}{ssfletters}{'007}
\DeclareMathSymbol{\bsfPhi}{0}{bsfletters}{'010}
\DeclareMathSymbol{\ssfPhi}{0}{ssfletters}{'010}
\DeclareMathSymbol{\bsfPsi}{0}{bsfletters}{'011}
\DeclareMathSymbol{\ssfPsi}{0}{ssfletters}{'011}
\DeclareMathSymbol{\bsfOmega}{0}{bsfletters}{'012}
\DeclareMathSymbol{\ssfOmega}{0}{ssfletters}{'012}

\newcommand{\fxfm}{\stackrel{\mathcal{F}}{\longleftrightarrow}}
\newcommand{\lxfm}{\stackrel{\mathcal{L}}{\longleftrightarrow}}
\newcommand{\zxfm}{\stackrel{\mathcal{Z}}{\longleftrightarrow}}

\DeclareMathOperator*{\gltop}{\gtreqless}
\newcommand{\glt}{\;\gltop^{\Hh=\svH_1}_{\Hh=\svH_0}\;}
\newcommand{\glty}{\;\gltop^{\Hh(\svy)=\svH_1}_{\Hh(\svy)=\svH_0}\;}
\newcommand{\gltby}{\;\gltop^{\Hh(\svby)=\svH_1}_{\Hh(\svby)=\svH_0}\;}
\DeclareMathOperator*{\geltop}{\genfrac{}{}{0pt}{}{\ge}{<}}
\newcommand{\gelty}{\;\geltop^{\Hh(\svy)=\svH_1}_{\Hh(\svy)=\svH_0}\;}
\newcommand{\geltby}{\;\geltop^{\Hh(\svby)=\svH_1}_{\Hh(\svby)=\svH_0}\;}
\renewcommand{\pe}{\Pr(e)}
\renewcommand{\defeq}{\triangleq}
\newcommand{\like}{\svlike}
\newcommand{\rvlike}{\mathssf{L}}
\newcommand{\sst}{\cl}
\newcommand{\svlike}{L}
\newcommand{\llike}{\rvllike}
\newcommand{\rvllike}{\cl}
\newcommand{\svllike}{l}
\newcommand{\bllike}{\rvbllike}
\newcommand{\rvbllike}{\boldsymbol{\cl}}
\newcommand{\svbllike}{\mathbf{l}}
\newcommand{\Qb}{\overline{Q}}
\renewcommand{\comb}[2]{\binom{#1}{#2}}


%% Random/sample variable/vector declarations.  Please add in alphabetical
%% order.  First section is for capitals.  Second for lower case.
% Capitals
\newcommand{\rvA}{{\mathssf{A}}}	% A
\newcommand{\svA}{A}
\newcommand{\rvbA}{{\mathbsf{A}}}
\newcommand{\svbA}{{\mathbf{A}}}
\newcommand{\rvFh}{{\hat{\mathssf{F}}}}	% F
\newcommand{\rvF}{{\mathssf{F}}}
\newcommand{\rvHh}{{\hat{\mathssf{H}}}}	% H
\newcommand{\rvH}{{\mathssf{H}}}
\newcommand{\svH}{H}
\newcommand{\svHh}{{\hat{\svH}}}
\newcommand{\rvL}{{\mathssf{L}}}	% L
\newcommand{\rvN}{{\mathssf{N}}}	% N
\newcommand{\rvR}{{\mathssf{R}}}	% R
\newcommand{\rvRh}{{\hat{\rvR}}}
\newcommand{\rvS}{{\mathssf{S}}}	% S
\newcommand{\rvSh}{{\hat{\rvS}}}
\newcommand{\rvW}{{\mathssf{W}}}	% W
\newcommand{\rvX}{{\mathssf{X}}}  	% X, random variable
\newcommand{\svX}{X}		  	
\newcommand{\rvXt}{{\tilde{\rvX}}}
\newcommand{\rvY}{{\mathssf{Y}}}	% Y
\newcommand{\rvZ}{{\mathssf{Z}}}	% Z

\newcommand{\rva}{{\mathssf{a}}}	% a
\newcommand{\rvah}{{\hat{\rva}}}
\newcommand{\sva}{a}
\newcommand{\svah}{{\hat{\sva}}}
\newcommand{\rvba}{{\mathbsf{a}}}
\newcommand{\svba}{{\mathbf{a}}}
\newcommand{\rvb}{{\mathssf{b}}}	% b
\newcommand{\rvc}{{\mathssf{c}}}	% c
\newcommand{\rvd}{{\mathssf{d}}}	% d
\newcommand{\rvbd}{{\mathbsf{d}}}	% d
\newcommand{\rve}{{\mathssf{e}}}	% e
\newcommand{\sve}{e}
\newcommand{\rvbfe}{{\mathbsf{e}}}
\newcommand{\svbfe}{{\mathbf{e}}}
\newcommand{\rvf}{{\mathssf{f}}}	% f
\newcommand{\svf}{f}
\newcommand{\rvbff}{{\mathbsf{f}}}
\newcommand{\svbff}{{\mathbf{f}}}
\newcommand{\rvh}{{\mathssf{h}}}	% h
\newcommand{\rvk}{{\mathssf{k}}}	% k
\newcommand{\svk}{k}
\newcommand{\rvm}{{\mathssf{m}}}	% m
\newcommand{\rvn}{{\mathssf{n}}}	% n
\newcommand{\rvbn}{{\mathbsf{n}}}
\newcommand{\rvq}{{\mathssf{q}}}	% q
\newcommand{\svq}{q}
\newcommand{\rvr}{{\mathssf{r}}}	% r
\newcommand{\svr}{r}
\newcommand{\rvs}{{\mathssf{s}}}	% s
\newcommand{\svs}{s}
\newcommand{\rvt}{{\mathssf{t}}}	% t
\newcommand{\svt}{t}
\newcommand{\rvu}{{\mathssf{u}}}	% u
\newcommand{\svu}{u}
\newcommand{\svuh}{{\hat{\svu}}}
\newcommand{\rvbu}{{\mathbsf{u}}}
\newcommand{\svbu}{{\mathbf{u}}}
\newcommand{\rvv}{{\mathssf{v}}}	% v
\newcommand{\svv}{v}
\newcommand{\svvh}{{\hat{\svv}}}
\newcommand{\rvbv}{{\mathbsf{v}}}
\newcommand{\svbv}{{\mathbf{v}}}
\newcommand{\rvvh}{{\hat{\rvv}}}
\newcommand{\rvw}{{\mathssf{w}}}	% w
\newcommand{\svw}{w}
\newcommand{\rvwh}{{\hat{\rvw}}}
\newcommand{\svwh}{{\hat{\svw}}}
\newcommand{\rvbw}{{\mathbsf{w}}}
\newcommand{\svbw}{{\mathbf{w}}}
\newcommand{\rvx}{{\mathssf{x}}}	% x, random variable
\newcommand{\rvxh}{{\hat{\rvx}}}
\newcommand{\rvxt}{{\tilde{\rvx}}}
\newcommand{\svx}{x}			% sample value
\newcommand{\svxh}{{\hat{\svx}}}
\newcommand{\svxt}{{\tilde{\svx}}}
\newcommand{\rvbx}{{\mathbsf{x}}}
\newcommand{\rvbxh}{{\hat{\rvbx}}}
\newcommand{\rvbxt}{{\tilde{\rvbx}}}
\newcommand{\svbx}{{\mathbf{\svx}}}
\newcommand{\svbxt}{{\tilde{\svbx}}}
\newcommand{\svbxh}{{\hat{\mathbf{x}}}}
\newcommand{\rvy}{{\mathssf{y}}}	% y
\newcommand{\rvyh}{{\hat{\mathssf{y}}}}
\newcommand{\svy}{y}
\newcommand{\rvyt}{{\tilde{\rvy}}}
\newcommand{\svyt}{{\tilde{\svy}}}
\newcommand{\svyh}{{\hat{\svy}}}
\newcommand{\rvby}{{\mathbsf{y}}}
\newcommand{\rvbyt}{{\tilde{\rvby}}}
\newcommand{\svby}{{\mathbf{y}}}
\newcommand{\svbyt}{{\tilde{\svby}}}
\newcommand{\rvz}{{\mathssf{z}}}	% z
\newcommand{\rvzh}{{\hat{\rvz}}}
\newcommand{\rvzt}{{\tilde{\rvz}}}
\newcommand{\svz}{z}
\newcommand{\svzh}{{\hat{\svz}}}
\newcommand{\rvbz}{{\mathbsf{z}}}
\newcommand{\svbz}{{\mathbf{z}}}

% vectors

\newcommand{\bzero}{{\mathbf{0}}}
\newcommand{\bOne}{{\mathbf{1}}}

% Handle uppercase Greek differently
\newcommand{\rvTh}{\ssfTheta}
\newcommand{\svTh}{\Theta}
\newcommand{\rvbTh}{\bsfTheta}
\newcommand{\svbTh}{\boldsymbol{\Theta}}
\newcommand{\rvPh}{\ssfPhi}
\newcommand{\svPh}{\Phi}
\newcommand{\rvbPh}{\bsfPhi}
\newcommand{\svbPh}{\boldsymbol{\Phi}}

\newcommand{\ddx}{\frac{\p}{\p \svx}}
\newcommand{\ddbx}{\frac{\p}{\p\svbx}}

%  --add new macros below this line--
\newcommand{\ybar}{\ensuremath{\bar{y}}}
\newcommand{\xbar}{\ensuremath{\bar{x}}}

\newcommand{\gauss}{\mathtt{N}}
\newcommand{\bern}{\mathtt{B}}
\newcommand{\unif}{\mathtt{U}}
\newcommand{\poiss}{\mathtt{P}}
\newcommand{\expfam}{\mathtt{E}}

\newcommand{\kron}{\mathbbb{1}}
